\documentclass[10pt,titlepage,twoside,a4paper]{report}
\usepackage[margin=1in,a4paper]{geometry}
\usepackage[italian]{babel}
\usepackage{lmodern}
\usepackage[latin1]{inputenc}
\usepackage{graphics,url,amsfonts,epsfig,hyperref,minted,float,listings}
\usepackage[usenames,dvipsnames,svgnames,table]{xcolor}
\usepackage{setspace}
    \doublespacing

%%%%%% Enable to debug
%\usepackage{showframe}


\definecolor{bg}{rgb}{0.95,0.95,0.95}

% Global minted options.
\usemintedstyle{vim}
% Verbatim imput
\newminted{text}{fontsize=\footnotesize,frame=single,framesep=10pt,breaklines,bgcolor=bg,linenos=true}
% Inputfiles
\newmintedfile[bashcode]{bash}{fontsize=\footnotesize,frame=single,framesep=10pt,breaklines,bgcolor=bg,linenos=true}



\title{Integrazione di una applicazione web con strumenti per la statistica}
%\title{\textsc{Integrazione di una applicazione web con strumenti per la statistica}}
\author{Franco Masotti}

\begin{document}
\pagenumbering{roman}

\maketitle
\newpage
\tableofcontents
\newpage
\listoffigures
\newpage
\listoftables
\newpage
\listoflistings
\cleardoublepage

\pagenumbering{arabic}

\chapter{Indice provvisorio} \label{Indice provvisorio}
L'oggetto di questa tesi \'e spiegare come \'e avvenuta l'integrazione di 
componenti per la statistica per l'applicazione "SWISH" in particolare
per quel che riguarda
\begin{itemize}
    \item l'interfacciamento con R,
    \item l'adattamento dell'installazione per R,
    \item ed infine la creazione di grafici per R con l'aggiunta di esempi.
\end{itemize}

Possibili capitoli:
\begin{enumerate}
    \item Introduzione (da fare alla fine)
    \item Scopo???
    \item Ambiente di lavoro
    \begin{enumerate}
        \item R
        \item SWI Prolog and SWISH
        \item QEMU
        \item Distribuzioni
            \begin{enumerate}
                \item Antergos + Parabola
                \item Debian + Trisquel
                \item speigare perch\'e 
                      l'utilizzo di distro FSDG e non mainstrewam (free sw + 
                      super updated (Parabola)).
            \end{enumerate}
        \item Perch\'e in macchina virtuale piuttosto che direttamente sul PC
        \item Mio script qvm semplifica l'uso di qemu attraverso l'utilizzo di 
              backup del VHD, etc...
    \end{enumerate}
    \item Pacchetti Arch Linux e derivate + Debian e derivate
    \begin{enumerate}
        \item Risolve il problema di "Adattare l'installazione per R" 
              (rserve-sandbox-docker) oltre che fornire un modo semplice per 
              installare l'interfaccia web (swish-cplint).
        \item Scrittura del pacchetto swish da swish-cplint
        \item Linee guida da apllicare a tutte le distribuzioni: perch\'e e 
              come funzionano.
        \item pacchetti Debian e Ubuntu-like (Trisquel). Spiegare perch\'e \'e 
              pi\'u fifficile fare qualcosa per questi sistemi (pacchetti 
              obsoleti).
        \item pacchetti bin: nel caso non si riesca a compilare/scariare 
              qualcosa.
    \end{enumerate}
    \item Libreria cplint\_r
    \begin{enumerate}
        \item Libreria prolog che si interfaccia con R e che consente il plot 
              dei gafici.
        \item Perch\'e ggplot2 (grammar of graphs)
        \item Metodo della libreria: Adattare i predicati della libreria cplint 
              ed adattarli alla nuova situazione, mantenendo i necessari e 
              creandone di nuovi (ovviamente solo quando necessario)
        \item Metodo degli esempi: copia file .pl in \_R.pl, adattamento per 
              usare la libreria, confronto con grafici originali in c3js
        \item <cplint\_r include cplint direttamente perche' in cplint c'\'e la 
              parte dei conti matematici (ancora valida) mentre la parte di plot 
              \'e presente in ogni esempio, quindi i predicati di cplint si 
              possono chiamare direttamente facendo poche modifiche.
        \item Cose degne di nota: Come funziona il passaggio di valori da 
              prolog ad R (e viceversa): speigare gli helpers che ho scritto. 
    \end{enumerate}
    \item Esempi prolog slegati da cplint\_r
    \begin{enumerate}
        \item Quali (gpr\_R.pl esempio pi\'u difficile da trasformare).
        \item Perch\'e: es: perch\'e non si poteva usare cplint\_r
        \item Come
    \end{enumerate}
    \item Documentazione
    \begin{enumerate}
        \item Perch\'e in texinfo: perch\'e permette di creare output in vari 
              formati automaticamente.
    \end{enumerate}
    \item Conclusione
    \begin{enumerate}
        \item Imparato rudimenti programmazione logica
        \item Imparato a costruire strutture semplici per ottenere risultati 
              "complessi" (Makefile dei pacchetti, stessa cplint\_r).
    \end{enumerate}
\end{enumerate}

\chapter{Ambiente di lavoro} \label{Ambiente di lavoro}
Segue una lista dei pi\'u importanti software utilizzati. \'E da notare che 
tutti sono software libero.

\section{SWISH ed SWI Prolog}
La piattaforma \emph{SWISH}\cite{SWISH} viene usata per la programmazione 
logica attraverso un qualunque web browser dotato di Javascript. Questa 
\'e collegata direttamente all'interprete SWI prolog ed \'e accessibile da 
shell con il comando \texttt{swipl}.

\section{Cplint on SWISH}
\emph{Cplint on SWISH} \'e una versione modificata di SWISH che include la 
libreria \emph{cplint}, la quale permette di formulare inferenze e 
compiere processi di apprendimento\cite{cplint}.

\section{Il linguaggio R}
\emph{R} \'e un ambiente software dedicato alla statistica e 
alla rappresentazione grafica di dati\cite{R}.

\section{ggplot2}
\emph{ggplot2} \'e un pacchetto R che permette di fare plot usando la 
cosiddetta \emph{grammatica dei grafici}, cio\'e la divisione per contesti dei 
vari aspetti di un grafico (check this...)


\chapter{Preparazione}
Per completare tutto il lavoro e testare le modifiche sui programmi man 
mano, \'e stato necessario installare l'intera piattaforma sul computer. 
Questo \'e stato possibile attraverso l'utilizzo di macchine virtuali 
per ragioni di comodit\'a.

\section{Macchine virtuali}
Per agevolare l'installazione ed il mantenimento della piattaforma web su una 
istanza locale si sono utilizzate macchine virtuali con immagini di 
distribuzioni GNU/Linux diverse quali: Antergos, Parabola GNU/Linux-libre, 
Debian e Trisquel.

\subsection{Qemu Virtual Machine}
Utilizzando il programma \emph{QEMU}\cite{QEMU} si \'e potuto scrivere uno 
programma per 
shell, \emph{qvm}, in modo da gestire facilmente tutti i casi d'uso che si 
sono presentati\cite{qvm}. Questo si \'e rivelato molto utile per esempio per 
gestire i backup  dei dischi rigidi virtuali. Per testare nuovi 
pacchetti oppure nel caso che qualche componente del sistema si fosse 
danneggiato. Sono infatti sufficienti un paio di comandi per ripristinare il 
backup e corregere l'errore.

Lo script d\'a anche la possibilit\'a di condividere file fra la macchina host 
e guest attraverso una semplice directory.

Poich\'e l'interfaccia web di SWISH \'e accessibile attraverso un port ben 
definito, si \'e fatto in modo che tale port fosse accessibile dalla macchina 
host. Oltre a questo \'e anche possibile l'accesso attraverso attraverso SSH. 
Mettendo insieme queste due caratteristiche non \'e neccessario avviare 
l'interfaccia grafica. Di seguito \'e riportata l'opzione di aiuto mentre lo 
script si trova  cite appendix A.

\begin{listing}[H]
\begin{textcode*}{}
Usage: qvm [OPTION]
Trivial management of 64 bit virtual machines with qemu.

Options:
    -a, --attach                connect via SSH
    -b, --backup                backup vhd
    -c, --create                create new vhd
    -d, --delete                delete vhd backup
        --delete-orig           delete original vhd
    -h, --help                  print this help
    -i, --install               install img on vhd
    -n, --run-nox               run vm without opening a graphical window
                                (useful for background jobs like SSH)
        --run-nox-orig          run-orig and run-nox combined
    -s, --mkdir-shared          create shared directory
    -x, --run                   run vm
        --run-orig              run from original vhd


Only a single option is accepted.
By default, the backup vhd is run.

CC0
Written in 2016 by Franco Masotti/frnmst <franco.masotti@student.unife.it>
\end{textcode*}
\caption{Pagina di aiuto di qvm}
\end{listing}

\subsection{Installazione e avvio di una nuova macchina virtuale con qvm}
La prima cosa da compiere \'e quella di ottenere l'immagine \emph{ISO} della 
distribuzione da installare.

Successivamente si deve configurare il file \texttt{configvmrc} inserendo il 
nome del file ISO.

Poi \'e necessario eseguire \mint{bash}|$ ./qvm -c| per creare un nuovo 
disco rigido virtuale, poi \mint{bash}|$ ./qvm -i| per l'installazione.

Una volta terminata l'installazione si pu\'o aggiungere SSH sulla macchina 
guest oltre che aggiungere la seguente riga nel file \texttt{/etc/fstab} della 
macchina virtuale in modo da aggiungere la directory condivisa:

\begin{listing}[H]
\begin{textcode*}{}
host_share /home/<shared_directory_path> 9p noauto,x-systemd.automount,trans=virtio,version=9p2000.L 0 0
\end{textcode*}
\caption{Comando fstab}
\end{listing}

Infine si avvia la macchina virtuale

\chapter{Scrittura dei pacchetti per le varie distribuzioni} \label{Scrittura 
dei pacchetti per le varie distribuzioni}
Per affrontare il problema dell'installazione di R si \'e reas necessaria la 
creazione di vari pacchetti.

...

% Appendici.
\appendix
\chapter{qvm}
\bashcode{qvm}
\bashcode{configvmrc}

\chapter{smt}
hi

\bibliographystyle{plain}
% file names without the extension, comma separated.
\bibliography{ref}

\end{document}
