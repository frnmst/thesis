\documentclass[12pt,titlepage]{book}
\usepackage[italian]{babel}
\usepackage{graphics}
\usepackage{url,amsfonts,epsfig}
\usepackage[latin1]{inputenc}


\title{\textsc{Integrazione di una applicazione web con strumenti per la statistica }}
\author{Franco Masotti}

\begin{document}
\pagenumbering{roman}

%%%% Opzione per interlinea 2
\baselineskip 18pt

\maketitle

\tableofcontents
\listoffigures
\listoftables

\pagenumbering{arabic}

\chapter{Indice provvisorio} \label{Indice provvisorio}
Il compito del tirocinio \'e stato il seguente:

\begin{itemize}
    \item Interfacciamento con R
    \item Adattare l'installazione per R
    \item Creazione di grafici per R con l'aggiunta di esempi.
\end{itemize}

Possibili capitoli:
\begin{enumerate}
    \item Introduzione (da fare alla fine)
    \item Scopo???
    \item Ambiente di lavoro
    \begin{enumerate}
        \item R
        \item SWI Prolog
        \item QEMU
        \item Distribuzioni
            \begin{enumerate}
                \item Antergos + Parabola
                \item Debian + Trisquel
                \item speigare perch\'e 
                      l'utilizzo di distro FSDG e non mainstrewam (free sw + 
                      super updated (Parabola)).
            \end{enumerate}
        \item Perch\'e in macchina virtuale piuttosto che direttamente sul PC
        \item Mio script qvm semplifica l'uso di qemu attraverso l'utilizzo di 
              backup del VHD, etc...
    \end{enumerate}
    \item Pacchetti Arch Linux e derivate + Debian e derivate
    \begin{enumerate}
        \item Risolve il problema di "Adattare l'installazione per R" 
              (rserve-sandbox-docker) oltre che fornire un modo semplice per 
              installare l'interfaccia web (swish-cplint).
        \item Scrittura del pacchetto swish da swish-cplint
        \item Linee guida da apllicare a tutte le distribuzioni: perch\'e e 
              come funzionano.
        \item pacchetti Debian e Ubuntu-like (Trisquel). Spiegare perch\'e \'e 
              pi\'u fifficile fare qualcosa per questi sistemi (pacchetti 
              obsoleti).
        \item pacchetti bin: nel caso non si riesca a compilare/scariare 
              qualcosa.
    \end{enumerate}
    \item Libreria cplint\_r
    \begin{enumerate}
        \item Libreria prolog che si interfaccia con R e che consente il plot 
              dei gafici.
        \item Perch\'e ggplot2 (grammar of graphs)
        \item Metodo della libreria: Adattare i predicati della libreria cplint 
              ed adattarli alla nuova situazione, mantenendo i necessari e 
              creandone di nuovi (ovviamente solo quando necessario)
        \item Metodo degli esempi: copia file .pl in \_R.pl, adattamento per 
              usare la libreria, confronto con grafici originali in c3js
        \item <cplint\_r include cplint direttamente perche' in cplint c'\'e la 
              parte dei conti matematici (ancora valida) mentre la parte di plot 
              \'e presente in ogni esempio, quindi i predicati di cplint si 
              possono chiamare direttamente facendo poche modifiche.
        \item Cose degne di nota: Come funziona il passaggio di valori da 
              prolog ad R (e viceversa): speigare gli helpers che ho scritto. 
    \end{enumerate}
    \item Esempi prolog slegati da cplint\_r
    \begin{enumerate}
        \item Quali (gpr\_R.pl esempio pi\'u difficile da trasformare).
        \item Perch\'e: es: perch\'e non si poteva usare cplint\_r
        \item Come
    \end{enumerate}
    \item Documentazione
    \begin{enumerate}
        \item Perch\'e in texinfo: perch\'e permette di creare output in vari 
              formati automaticamente.
    \end{enumerate}
    \item Conclusione
    \begin{enumerate}
        \item Imparato rudimenti programmazione logica
        \item Imparato a costruire strutture semplici per ottenere risultati 
              "complessi" (Makefile dei pacchetti, stessa cplint\_r).
    \end{enumerate}
\end{enumerate}

a\cite{one2017,two2018}


%%% OBBLIGATORIA:

\bibliographystyle{plain}
% file names without the extension, comma separated.
\bibliography{ref}

\end{document}
